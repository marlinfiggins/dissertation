\chapter{Conclusion}

% TODO: Frame the overarching challenge: understanding and forecasting SARS-CoV-2 variant dynamics by combining statistical models, mechanistic insights, and molecular data.

% TODO: Highlight the unifying principle of the thesis: mechanistic and biological structure, combined with statistical rigor, enhances forecast reliability and enables predictive power.

% TODO:Position the process of evaluation, mechanistic integration, and ... as the central driver of the thesis.

% TODO: Acknowledge remaining challenges, such as global inequities in data availability and limitations in predicting novel variants.

This dissertation establishes evolutionary forecasting as a rigorous and dynamic process by integrating insights from population genetics, mathematical epidemiology, and statistics. 
Through systematic evaluation, this work identifies gaps in existing models and lays a foundation for addressing them by integrating mechanistic insight with immunological, genetic, and epidemiological data.
Forecasting moves beyond tracking trends to become a dynamic process that unites real-time analysis, biological insight, and long-term evolutionary prediction. These advances rely on domain knowledge, statistical methodology, and data on the biological and environmental forces shaping pathogen evolution. They enable new kinds of predictions, such as variant-specific $R_t$, pseudo-immune representations, and epidemic growth rates derived from selective pressure, enhancing our ability to anticipate and respond to pathogen evolution.

This work presents several key contributions—mechanistic fitness models, forecasting evaluation methods, and data-informed fitness predictions—that deepen our understanding of pathogen evolution.
These contributions are operationalized through tools like \evofr\ and \forecastsNcov.
\evofr\ serves as a flexible toolkit for real-time analysis and forecasting while \forecastsNcov\ automates these methods to forecast in real-time, transforming methodological detail into accessible, reproducible insights for public health decision-makers.
Together, these contributions interpret today’s dynamics while predicting the evolution of tomorrow, turning theoretical advancements into modular, scalable systems capable of addressing diverse epidemiological and evolutionary challenges.

By emphasizing modularity and scalability, this work ensures its contributions extend beyond SARS-CoV-2 to other pathogens such as influenza or emerging zoonotic threats.
This adaptability, combined with interdisciplinary insights from genomics, immunology, epidemiology, computational biology, and statistics, strengthens our approach by ensuring it is both biologically meaningful and statistically robust.
These contributions provide the scientific and practical tools scientists and public health systems need to understand, interpret, and act on the dynamics of emerging genetic variants.

% TODO: Future work:

This dissertation advances evolutionary forecasting by integrating mechanistic insights and empirical data into models and tools that predict the dynamics of viral evolution. While significant progress has been made, the field presents opportunities for further development, particularly in capturing the complexity of viral evolution and ensuring the practical utility of forecasts.

A critical avenue for future research is expanding the scope of data incorporated into forecasting models. While this work demonstrates the power of genomic surveillance, viral evolution is influenced by behavioral, clinical, and environmental factors that are not yet fully integrated.
Incorporating these dimensions would enhance the selective pressure metrics and fitness estimation models developed here, offering a richer understanding of the factors shaping variant success and public health outcomes.

The methodologies presented in this dissertation, though tailored to SARS-CoV-2, suggest broader applicability to other rapidly evolving pathogens such as influenza and RSV.
Generalizing these models will require adapting them to the specific evolutionary and immunological dynamics of other viruses.
This extension would establish evolutionary forecasting as a universal framework for monitoring and predicting pathogen evolution.

In parallel, the development of generative sequence models offers a promising direction.
Such models could simulate evolutionary trajectories, predicting not only which variants may emerge but also their phenotypic traits. These capabilities would build on the fitness and immune escape dynamics studied here, enabling long-term forecasting and preemptive public health strategies.

Further refinement of population structure models is also essential.
Selective pressures and fitness advantages vary across populations due to differences in immunity, geography, and temporal dynamics.
Building on the latent factor models and selective pressure metrics developed in this dissertation, future work can better account for these variations, improving the precision of forecasts across diverse contexts.

On a practical level, this dissertation demonstrates how tools like \evofr and \forecastsNcov can operationalize evolutionary forecasting.
To maximize their impact, future efforts should focus on making these tools more accessible to public health practitioners through user-friendly interfaces and seamless integration into existing workflows.
Collaboration with immunologists, virologists, and public health experts will be critical for validating these models and ensuring their relevance in real-world decision-making.

Finally, as forecasting tools become increasingly influential in guiding vaccine updates and public health policies, their ethical and policy dimensions require careful consideration.
Future work should explore frameworks to ensure forecasts are used transparently and equitably, addressing issues such as resource allocation and public communication.

The next phase of evolutionary forecasting lies in refining its scientific foundations while expanding its practical applications. By integrating diverse data sources, adapting models to new pathogens, and addressing ethical concerns, the field can continue to evolve, meeting the needs of both science and society.

%TODO: Mention specifically longitudinal immune profiling data and that the reality is that data availability is declining and we need methods that can make due with what we have and suggest cost-effective means of expanding data collection

% TODO: Edit future work

In my opinion, this work offers a clear lesson beyond advancing forecasting or any single pathogen.
It has shown that pursuing a deeper understanding of dynamic systems and the structures that drive them creates new opportunities for innovation.
In that understanding lies the potential to anticipate, to prepare, and to act.
