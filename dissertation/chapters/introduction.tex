\chapter {Introduction}

Novel genetic variants are often produced due to mutations in currently circulating virus populations.
These mutations can sometimes provide fitness advantages to members of the population allowing them to out-compete other variants through several mechanisms such as partial immune escape, increased transmissibility, among others.
This interplay of mutation, transmission, and selection leads to evolution in the population.
Therefore, understanding the genetic composition of viral populations and its relation to virus phenotype can be useful for understanding the current and future epidemic potential of viral variants.

My interest is in modeling the epidemic and evolutionary dynamics of virus transmission jointly to better understand how and why virus populations change over time and the consequences of this on human health.

Ultimately, I would like to assess the potential for joint epidemic and evolutionary models to forecast the genetic composition of virus population in the future.
This may be particularly useful for vaccine strain selection for seasonal influenza and SARS-CoV-2.

To start, I will introduce several preliminary mathematical, biological, and statistical concepts and results needed to re-frame these aims as concrete goals.

\section{Epidemic models}

We'll give a basic introduction of the ordinary differential equation formulation of epidemic models. 
We begin with the classic SIR model \cite{KermackMcKendrick1927}.

We assume that there are three types of individuals in the population: susceptible ($S$), infected ($I$), and recovered $R$. 
Susceptible individuals can become infected, infected individuals eventually recover (or die), and recovered individuals cannot become infected once again. 
This gives us the following system of equations
\begin{align}
  \frac{d S}{d t} &= - \beta S I\\ 
  \frac{d I}{d t} &= \beta S I - \gamma I\\
  \frac{d R}{d t} &= \gamma I,
\end{align}
where here $\beta$ is an effective infectious contact rate, $\gamma$ is a recovery rate. 

This form of the model lends itself to a couple of analyses which will become important as we extend these models to more complicated infection dynamics.

\paragraph{Effective reproduction number}%

The first is the effective reproduction number $R(t)$. 
We can see that when $R_{t} = \beta S(t) / \gamma > 1$, the number of infected individuals will be increasing $\frac{dI}{dt} > 0$. 
This quantity $R_{t}$ controls the direction of the epidemic.
In the special case where the population is fully susceptible ($S(t) = 1$), this quantity is called the basic reproduction number $R_{0}$ and it is interpreted as the average number of secondary infections caused by a single individual in an otherwise completely susceptible (naive) population.

\paragraph{Infectious period}

The second analysis answers the question of how long individuals stay infected under this model. One can show that from the point of infection an individuals rate of becoming uninfectious is constant which implies that the total time spent infected follows an exponential distribution with mean $\tau = \frac{1}{\gamma}$.

Using these facts, we can re-write the differential equation for $I$ as
\begin{align}
  \frac{dI}{dt} &= \tau^{-1} (R_{t} - 1)I = r(t) I.
\end{align}
Recognizing the differential equation above as that of a time-inhomogeneous exponential growth. 
We notice that the timing of infectiousness and the effective reproduction number jointly determine the exponential growth rate in an epidemic. 
In particular, this ODE formulation corresponds to the assumption that the timing between these infections has an exponential distribution. 
This relationship between timing of reproduction and magnitude of reproduction can be shown more generally.
%TODO: Though there are ways to generalize this exponential distributed infectious period in the ODE framework, we'll opt for the more general age-structured population dynamic approach.
%TODO: Mention that generation time and R are needed to make sense of (persistence potential of) exponential growth
%TODO: Show that this is is not generally true.

\paragraph{Relating effective reproduction number and infection timing}%

We'll also consider more explicitly age-structured transmission models such as renewal equation based models.

In these models, we have that incidence follows
\begin{align}
  I(t) = R_{t} \int_{0}^{t} I(t-\tau) g(\tau) d\tau,
\end{align}
where $g$ is the density of the generation time $G$ i.e. the timing between primary and secondary infections which acts as a form of the relative transmissibility over the course of an infection.
We can derive the $R_{t}$ from an age structured population with exponential growth rate $r$ and generation time $G$ as
\begin{align}
  \frac{1}{R_{t}} &= \int_{\bbR^{+}} e^{-r\tau} g(\tau) d\tau\\
                  &=\mathcal{L}_{G}(r) = M_{G}(-r),
\end{align}
where we've written the above in terms of a Laplace transform $\mathcal{L}_{G}$ or the moment generating function of $M_{G}$ of the generation time.
This relationship between $r$ and $R$ can be computed in closed form for several distributions such as the exponential and Gamma \cite{Wallinga2006}.
For general generation times, we can re-write this in terms of the cumulant generating function of $G$ $K_{G}$ to give the following relationship in terms of moments of $G$, $r$, and $R_{t}$
\begin{align}
  \log R_{t} = &- K(-r) \approx \Expect[G] r - \Var[G] \frac{r^{2}}{2} + \left( \Var[G]^{3 / 2} \text{Skew}[G]\right) \frac{r^{3}}{6} + o(r^{4}). \label{eq:central_moment_R}
\end{align}

Here, it becomes clear that timing of infection and infectiousness is important for interpreting exponential growth rates in terms of the ``average'' transmission dynamics or $R_{t}$ in a population.

% We can partially remedy this within the ODE approach by using the "chain trick" in which we allow the infectious individual to remain infectious for non-exponentially generated time period by having multiple sequentially-chained infectious compartments with recover rates $n\gamma$, meaning we replace
% \begin{align*}
%   \frac{d I_{1}}{d t}  &= \beta S \sum_{k=1}^{n} I_{k} - \gamma I_{1}\\
%   \frac{d I_{i}}{d t}  &= n\gamma I_{i-1}- n\gamma I_{i}, \quad 2\leq i \leq n\\
%   \frac{d R}{dt} &= n \gamma I_{n}
% \end{align*}
% In this case, the infectious period is $\text{Gamma}(n, n\gamma)$ distributed with mean $\gamma^{-1}$ and variance. This follows as the infectious period is now the sum of $n$ IID $\text{Exp}(n\gamma)$ random variables. 

% Large $n$ required for complicated serial intervals. Is there a timing first approach.

% More complex models. Specifically allowing us to test various mechanisms for transmission

% Computing $R$ and $R_{0}$. Reference next generation matrix approach here?


% Underlying assumptions of these models with respect to timing on infections. Chain trick as remedy here, reference data showing that these serial intervals are non-exponential to satisfy Mark?
% R, Rt, and methods for estimating R from incidence data?

\section{Multistrain models}

We'll now dive further into the evolutionary component of these models. 
In order for a evolution to occur, we need a counter-balance of mutation (generating heritable diversity) and selection (differential survival and reproduction based on that diversity).

For the goal of evolutionary forecasting of virus strains, we need to develop a notion of a multistrain model wherein there are different types of infections and in which these differences are heritable and in which there can be selection for different strains.

\paragraph{A two-strain epidemic model}%

We first consider one of the most standard ODE-based model for infectious diseases with multiple variants: an SIR model in which there are a wild-type $\text{wt}$ and a variant $\text{var}$. 
We assume that the variant virus can differ in two ways from the wild type: it may have differ in innate transmissibility by a factor of $\eta_{T}$ or it may differ in its infectious period by a factor $\eta_{G}$. 
This gives us the following system of ODEs:
\begin{align}
    \frac{d S}{d t} &= - \beta S I_{\text{wt}} - \beta \eta_{T} S I_{\text{var}},\\
  \frac{d I_{\text{wt}} }{d t} &= \beta S I_{\text{wt}} - \gamma I_{\text{wt}} = r_{\text{wt}}(t) I_{\text{wt}},\\
  \frac{d I_{\text{var}}}{d t} &= \beta \eta_{T} S I_{\text{var}} - \gamma \eta_{G} I_{\text{var}} = r_{\text{var}}(t) I_{\text{var}}.
\end{align}

In this model, we know have two different $R_{0}$ values which are $R_{0}^{\text{wt}} = \frac{\beta}{\gamma}$ and $R_{0}^{\text{var}} = \frac{\eta_{T}}{\eta_{G}}R_{0}^{\text{wt}}$.

Though this model is not appropriate for infectious diseases with non-trivial competition dynamics via strain-specific immunity for example, it gives us an essential result for epidemiological modeling of genetic variants.

We see that once again there is a dependence between transmissibility and the timing of infection. 
More precisely, we notice that without specifying $\eta_{G}$, the variant transmission advantage is unidentifiable. 

\section{Approaches in population genetics}%

Traditionally, population genetic methods are used to identify selective advantages of genetic variants. 
These models often work on the level of frequency of genetic variants and are concerned with estimating the selection coefficient $s$ (relative fitness difference) of genetic variants.

One model in this vein that has gained popularity for analyzing SARS-CoV-2 variant frequencies and estimating relative fitness is multinomial logistic growth (MLR).
Assuming that each variant has a fixed relative fitness to all others, we can derive the following model for variant frequencies over time
\begin{align*}
f_{v}(t) = \frac{p_{v}\exp(r_{v}t)}{\sum_{u} p_{u} \exp(r_{u}t)},
\end{align*}
where $f_{v}$ is the frequency of variant $v$ in the virus population at time $t$.
Here, $r_{v}$ and $p_{v}$ can be thought of as variant-specific growth rates and initial prevalences. 
Due to the fact that variant frequencies must add to one, $r_{v}$ and $p_{v}$ are not uniquely identifiable for all variants. 
Therefore, we often instead work with a model of the form
\begin{align*}
f_{v}(t) = \frac{\exp(\beta_{v} t + \alpha_{v})}{\sum_{u} \exp(\beta_{u} t + \alpha_{u})},
\end{align*}
where we've set some variant as a pivot with $\beta_{\text{pivot}} = \alpha_{\text{pivot}}=0$.
Using these MLR models, the relative growth advantage of variant $v$ relative to the pivot is often interpreted as being $\beta_{v} \tau$ where $\tau$ is the mean generation time for the population.
This method has also been used to get approximations of the relative effective reproduction number of variants as $R_{t}^{v} / R_{t}^{\text{pivot}} = \exp(\beta_{v}\tau)$ which is equivalent to assuming that the generation time distribution is a point mass at $\tau$.

\section{Bayesian inference}%

For this general project of relating epidemic dynamics and population genetics, we need to take these models and connect them to what is observed or observable for statistical analysis.
In our case, we rely primarily on Bayesian inference methods.
We'll shortly review several features of Bayesian inference which are useful for this kind of modeling.

In short, Bayesian inference is focused on the properties of the joint probability distribution  $p(\vec{x}, \vec{\theta})$ of data $\vec{x}$ and parameters $\vec{\theta}$ under a given model $m$.

\paragraph{Posterior distribution.}
Given a model and data, we can provide estimates of a possible distribution of parameters conditional on observed data using Bayes rule
\begin{align*}
p(\vec{\theta} \mid \vec{x}) = \frac{p(\vec{x} \mid \vec{\theta})\cdot p(\vec{\theta})}{\int p(\vec{x} \mid \vec{\theta}') p(\vec{\theta}') d \vec{\theta}'}.
\end{align*}

This distribution is called the \emph{posterior distribution of $\theta$ under model $m$}. 
Notice, this distribution depends on both the likelihood of the data given the model parameters $\vec{\theta}$ and the prior probability of the parameters $p(\vec{\theta})$ before observing data.
These priors are often set as a part of model building and can be useful for regularization, smoothing, and modeling hierarchical data.

\paragraph{Posterior samples allow us to estimate expectations.}%

Most often, we approximate this distribution using samples from the posterior distribution using methods like Markov Chain Monte Carlo or Variational Inference.

These samples allow us to get expectations of function under the posterior distribution, so that we can compute
\begin{align*}
\Expect_{p(\vec{\theta} \mid \vec{x})}[f(\vec{\theta})] = \int_{\Theta} f(\vec{\theta}) p(\vec{\theta} \mid \vec{x}) d \vec{\theta}.
\end{align*}

These expectations can be used to compute the probability of events under the posterior distribution including distributions for missing or future data.

%MF: Cut section on posterior predictive distribution?
% \paragraph{Posterior predictive distribution.}%

% In addition to the posterior distribution, we're often also interested in the \emph{posterior predictive distribution} which gets the distribution of new data $\vec{x}_{\text{new}}$ under the posterior distribution of $\theta$ i.e. conditioned on the original data set $\vec{x}$
%  \begin{align*}
% p(\vec{x}_{\text{new}} \mid \vec{x}) = \int_{\Theta} p(\vec{x}_{\text{new}} \mid \vec{\theta}) \cdot p(\vec{\theta} \mid \vec{x}) d\vec{\theta} = \Expect_{p(\vec{\theta} \mid \vec{x})}[p(\vec{x}_{\text{new}} \mid \vec{\theta})].
% \end{align*}
% This distribution is useful for analysing how likely observed data is under a given a model.

Using these objects in Bayesian inference allows us to incorporate prior information about a system of interest, estimate parameters accounting for uncertainty in the model and observation, and build models that can be informed by various data sources simultaneously. 

\subsection{Components of evolutionary forecasting}%

%TODO: Expand this. 
%TB comment: It's hard for me to parse really what's entailed from the paragraph as it stands.

With these preliminaries in mind, we can identify several steps towards our goal of mechanism-informed evolutionary forecasting of viruses.

First, we must be able to quantify transmission of genetic variants and estimate relative fitness of variants.
This will allow us to get estimates of underlying fitness advantages of currently circulating viruses which we can use to better understand how individual variants transmit and therefore better understand the interplay between relative fitness and epidemic dynamics.

Second, we must be able to tie relative fitness to specific characteristics of genetic variants.
This means understanding the biological bases for relative fitness differences in variants.
For example, emergence of variants which have increased transmissibility or immune escape potential against past immunity will likely lead to different epidemic dynamics and therefore different evolutionary outcomes. 

Third, we must assess how and if population structure such as past exposure and immunity may affect relative fitness estimates at the present and in the future.
Due to the fact that relative fitness is defined with respect to the transmission potential at a given environment, changes in host populations like accumulation of immunity via infection can lead to changes in the relative fitnesses of variants over space and time.
Understanding which aspects of population structure are important in determining epidemic dynamics and relative fitness can then inform us of how relative fitnesses of variants will change in the future.

Using the relative fitness estimates and our knowledge of its determinants, we can provide probabilistic forecasts of relative fitness for variants.
We can then project forward these relative fitnesses to get estimates of genetic composition of future population usually in the form of variant frequencies.

The rest of the proposal will combine the ideas mentioned in this section to develop and plan individual projects towards the larger project of joint epidemic and evolutionary modeling towards evolutionary forecasting.
